\ChapterWithAuthor{Stivă}{Traian Mihai Danciu}
\section{Noțiuni introductive}

Stiva este ca un teanc de obiecte. Ea are $4$ operații principale:
\begin{enumerate}
    \item \textit{push(value)}: Adaugă \textit{value} pe vârful stivei.
    \item \textit{top()}: Spune care este valoarea de pe vârful stivei.
    \item \textit{pop()}: Scoate elementul de pe vârful stivei.
    \item \textit{empty()}: Spune dacă stiva este goală.
\end{enumerate}

\begin{observation}
Valorile vor fi returnate după regula \textit{LIFO}, adică \textit{last in, first out}.
\end{observation}

\section{Problema \href{https://kilonova.ro/problems/2001}{stiva}}

Această problemă ne cere să implementăm exact operațiile descrise mai sus. Aceasta este soluția:
\cpp{codes/stiva/stiva-de-mana.cpp}

Aceasta este soluția cu stiva din STL:
\cpp{codes/stiva/stiva-stl.cpp}

\section{Problema \href{https://www.youtube.com/watch?v=dQw4w9WgXcQ}{stiva\_max\_min}}
Da, aprob, faina problemă Traiane 